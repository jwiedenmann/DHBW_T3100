%!TEX root = ../../main.tex

\chapter{Theoretische Grundlagen}

Dieses Kapitel bildet die Grundlage für die folgenden Vergleiche und Evaluationen von Graph Visualisierungen. Es werden Graphen und Knowledge Graphen genauer erläutert, ein Datenformat zur Speicherung der Daten und zuletzt die verschiedenen Visualisierungsoptionen. 

\section{Graphen und Knowledge Graphen}



\section{RDF}

\section{Visualisierungstechniken}

Im Zeitalter der Digitalisierung sind Daten allgegenwärtig und werden in einem nie dagewesenen Ausmaß gesammelt. Es ist von entscheidender Bedeutung, diese Informationen effektiv begreifen und teilen zu können. In diesem Zusammenhang erweist sich die Datenvisualisierung als ein relevantes Feld, welches als Vermittler zwischen komplexen Datenmengen und dem menschlichen Verständnis fungiert. Durch die Transformation von rohen Daten in ein grafische Darstellungen vereinfacht die Datenvisualisierung den Zugang zu den Informationen. Sie erlaubt es den Beteiligten, Muster, Trends und Anomalien zu identifizieren, die in den Rohdaten verborgen sein könnten. Die Umwandlung von abstrakten Zahlen in visuelle Darstellungen fördert nicht nur das Verständnis, sondern unterstützt auch effektivere Entscheidungsprozesse. Es lässt sich nicht stark genug hervorheben, wie wichtig Datenvisualisierung ist. Sie macht Datenverständnis für alle zugänglich, ganz egal, wie viel oder wie wenig technisches Wissen der Benutzer hat.

In dem breiten Feld der Datenvisualisierung bildet die Visualisierung von Knowledge Graphen den Fokus der Untersuchungen in diesem Kapitel. Knowledge Graphen stellen nicht nur Daten dar, sie offenbaren ein komplexes Netzwerk von Verbindungen, das einzelnen Datenpunkten Kontext und Bedeutung verleiht. Hierbei sind sowohl die Details der Verbindungen, als auch die übergeordnete Struktur von hoher Relevanz. Einerseits erlaubt die Betrachtung einzelner Knoten und Kanten einen detaillierten Einblick in die Beziehungen zwischen den Elementen. Andererseits erlaubt die Visualisierung der größeren Graphenstruktur, einen Blick auf die globale Topologie der Daten. Hier kann ein Verständnis für die übergeordneten Phänomene, wie Cluster, Hierarchien und Muster geschaffen werden, die aus den einzelnen Datenpunkten entstehen. Dieser doppelte Fokus bereichert das Verständnis der Daten und ermöglicht es, nicht nur die unmittelbaren Verbindungen, sondern auch die breiteren Muster und Strukturen zu erkennen. Auf diese Weise dient die Visualisierung von Knowledge Graphen als ein mächtiges Werkzeug, um die Komplexität vernetzter Informationen zu navigieren. Gleichzeitig bringt der doppelte Fokus allerdings auch Herausforderungen für jedes Werkzeug, welches die Visualisierung von Knowledge Graphen zum Ziel hat.

In den folgenden Abschnitten werden die möglichen Diagrammarten und Visualisierungstechniken, welche sich für die Darstellungen von Knowledge Graphen eignen detailliert beschrieben und erläutert.

\subsection{Diagrammarten}

Bei der Analyse von Knowledge Graphen spielt die Visualisierung eine entscheidende Rolle. Sie fördert das Verständnis der komplexen Beziehungen zwischen den Entitäten und der Struktur der Daten. Nach der Analyse der möglichen Diagrammarten stechen Node-Link Diagramme und Adjazenzmatrizen als die vielversprechendsten Kandidaten für die Visualisierung heraus. Dieses Kapitel geht auf diese beiden  Diagrammtypen ein. Es werden ihre Eigenschaften hervorgehoben und ihre Vor- und Nachteile erläutert.

\subsubsection{Node-Link Diagramme}