%!TEX root = ../../main.tex

\chapter{Theoretische Grundlagen}

Dieses Kapitel bildet die Grundlage für die folgenden Vergleiche und Evaluationen von Graph Visualisierungen. Es werden Graphen und Knowledge Graphen genauer erläutert, ein Datenformat zur Speicherung der Daten und zuletzt die verschiedenen Visualisierungsoptionen. 

\section{Graphen und Knowledge Graphen}

\section{Resource Description Framework}

\section{Visualisierungstechniken}

Im Zeitalter der Digitalisierung sind Daten allgegenwärtig und werden in einem nie dagewesenen Ausmaß gesammelt. Es ist von entscheidender Bedeutung, diese Informationen effektiv begreifen und teilen zu können. In diesem Zusammenhang erweist sich die Datenvisualisierung als ein relevantes Feld, welches als Vermittler zwischen komplexen Datenmengen und dem menschlichen Verständnis fungiert. Durch die Transformation von rohen Daten in ein grafische Darstellungen vereinfacht die Datenvisualisierung den Zugang zu den Informationen. Sie erlaubt es den Beteiligten, Muster, Trends und Anomalien zu identifizieren, die in den Rohdaten verborgen sein könnten. Die Umwandlung von abstrakten Zahlen in visuelle Darstellungen fördert nicht nur das Verständnis, sondern unterstützt auch effektivere Entscheidungsprozesse. Es lässt sich nicht stark genug hervorheben, wie wichtig Datenvisualisierung ist. Sie macht Datenverständnis für alle zugänglich, ganz egal, wie viel oder wie wenig technisches Wissen der Benutzer hat.

In dem breiten Feld der Datenvisualisierung bildet die Visualisierung von Knowledge Graphen den Fokus der Untersuchungen in diesem Kapitel. Knowledge Graphen stellen nicht nur Daten dar, sie offenbaren ein komplexes Netzwerk von Verbindungen, das einzelnen Datenpunkten Kontext und Bedeutung verleiht. Hierbei sind sowohl die Details der Verbindungen, als auch die übergeordnete Struktur von hoher Relevanz. Einerseits erlaubt die Betrachtung einzelner Knoten und Kanten einen detaillierten Einblick in die Beziehungen zwischen den Elementen. Andererseits erlaubt die Visualisierung der größeren Graphenstruktur, einen Blick auf die globale Topologie der Daten. Hier kann ein Verständnis für die übergeordneten Phänomene, wie Cluster, Hierarchien und Muster geschaffen werden, die aus den einzelnen Datenpunkten entstehen. Dieser doppelte Fokus bereichert das Verständnis der Daten und ermöglicht es, nicht nur die unmittelbaren Verbindungen, sondern auch die breiteren Muster und Strukturen zu erkennen. Auf diese Weise dient die Visualisierung von Knowledge Graphen als ein mächtiges Werkzeug, um die Komplexität vernetzter Informationen zu navigieren. Gleichzeitig bringt der doppelte Fokus allerdings auch Herausforderungen für jedes Werkzeug, welches die Visualisierung von Knowledge Graphen zum Ziel hat.

In den folgenden Abschnitten werden die möglichen Diagrammarten und Visualisierungstechniken, welche sich für die Darstellungen von Knowledge Graphen eignen detailliert beschrieben und erläutert.

\subsection{Diagrammarten}

Bei der Analyse von Knowledge Graphen spielt die Visualisierung eine entscheidende Rolle. Sie fördert das Verständnis der komplexen Beziehungen zwischen den Entitäten und der Struktur der Daten. Nach der Analyse der möglichen Diagrammarten stechen Node-Link Diagramme und Adjazenzmatrizen als die vielversprechendsten Kandidaten für die Visualisierung heraus. Dieses Kapitel geht auf diese beiden  Diagrammtypen ein. Es werden ihre Eigenschaften hervorgehoben und ihre Vor- und Nachteile erläutert.

\subsubsection{Node-Link Diagramme}

Node-Link Diagramme sind die intuitivste Art der Graphdaten-Visualisierung. Sie bieten eine zugängliche und visuell  ansprechende Möglichkeit, die Vielschichtigkeit von Netzwerken darzustellen. In diesen Diagrammen werden Entitäten als Knoten abgebildet, die meistens durch Kreise dargestellt werden. Um unterschiedliche Typen von Entitäten oder deren Zustände zu kennzeichnen können allerdings auch andere Formen verwendet werden. Die Beziehungen zwischen diesen Entitäten werden durch Linien oder Pfeile dargestellt, wobei häufig verschiedene Stile oder Farben genutzt werden, um die Beschaffenheit oder Intensität dieser Verbindungen zu illustrieren. Die räumliche Positionierung der Knoten und Verbindungen in Node-Link Diagrammen kann die Lesbarkeit sowie die aus der Visualisierung gewonnenen Einsichten stark beeinflussen. Eine sinnvolle Anordnung der Knoten erlaubt es die spezifischen Charakteristiken des Netzwerks, wie beispielsweise Cluster zu betonen und diese erkennbar zu machen (\cite{nodelink:Basics}).

Der größte Vorteil von Node-Link Diagrammen besteht in ihrer Fähigkeit, abstrakte relationale Daten begreifbar zu machen. Sie ermöglichen es dem Betrachter Muster zu entdecken, Unregelmäßigkeiten zu erkennen und die Struktur des Netzwerks durch visuelle Untersuchung zu begreifen. Das macht sie unverzichtbar für jeden Einsatz, bei dem die Beziehungsdynamiken und die Struktur der Daten von Belangen sind. Ein Node-Link Diagramm kann beispielsweise in der Analyse sozialer Netzwerke auf einen Blick die Influencer aufzeigen, indem es die Knoten mit den meisten Verbindungen hervorhebt (\cite{nodelink:Basics}).

Allerdings nimmt die Effektivität von Node-Link Diagrammen mit der Größe und Komplexität des Graphen ab. In dicht vernetzten oder sehr großen Graphen kann die Übersichtlichkeit schnell abnehmen. Überlappende Verbindungen und Knoten führen oft zu einem Durcheinander, das die Visualisierung unklar und verwirrend macht. Dieses Problem kann durch interaktive Visualisierungstools reduziert werden. Diese Werkzeuge erlauben es Nutzern den Graphen dynamisch zu vergrößern, zu filtern und zu durchsuchen. Dennoch bleibt die Herausforderung der Verständlichkeit bei großen Graphen bestehen (\cite{nodelink:DynamicGraph}, S. 3). Dies erfordert eine sorgfältige Planung des Diagrammdesigns und den Einsatz von Vereinfachungstechniken, wie kraftbasierte Layouts und Kantenverschmelzung, um die Lesbarkeit zu verbessern.

\subsubsection{Adjazenzmatrizen}

Einen gänzlich anderen Ansatz zur Visualisierung von Graphdaten bilden die Adjazenzmatrizen. Sie bieten eine strukturierte und analytische Perspektive auf die Daten. Adjazenzmatrizen verwenden ein rasterbasiertes Layout und eignen sich daher ausgezeichnet für die Abbildung komplexer und großflächiger Netzwerke. Bei einer Adjazenzmatrix werden die Entitäten des Graphen sowohl auf der horizontalen als auch auf der vertikalen Achse einer quadratischen Matrix aufgeführt. Die Zellen innerhalb der Matrix werden gefüllt, um die Existenz einer Beziehung zwischen den entsprechenden Entitäten zu signalisieren. Diese Herangehensweise verwandelt die Vernetzung des Graphen in ein visuelles Muster aus besetzten und unbesetzten Zellen. Eine besetzte Zelle am Kreuzungspunkt der Zeile ii und Spalte jj signalisiert eine direkte Verbindung von Entität ii zu Entität jj.

(Beispiel)

Die Stärke von Adjazenzmatrizen liegt in ihrer Effizienz bei der kompakten Darstellung großer Graphen. Diese Eigenschaft mach sie besonders für Netzwerke geeignet, bei denen ein Node-Link Diagramm zu unübersichtlich wäre. Die gleichförmige Anordnung einer Matrix vereinfacht das Erkennen von Mustern, wie beispielsweise Clustern verwandter Knoten, die sich durch zusammenhängende Blöcke gefüllter Zellen abzeichnen, oder das Aufspüren isolierter Teilnetzwerke innerhalb eines umfangreicheren Graphen. Zudem eignen sich Adjazenzmatrizen aufgrund ihrer strukturierten Beschaffenheit hervorragend für quantitative Analysen und Berechnungen, ermöglichen Operationen wie die Matrixmultiplikation, um Wege im Graphen zu ermitteln oder andere strukturelle Merkmale zu analysieren.

\subsubsection{Tree Maps}

\subsection{Vergleiche der Diagrammarten}

\subsection{Auswahl einer Diagrammart}

\section{Vereinfachungstechniken}