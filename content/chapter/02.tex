%!TEX root = ../../main.tex

\chapter{Theoretische Grundlagen}

Dieses Kapitel bildet die Grundlage für die folgenden Vergleiche und Evaluationen von Graph Visualisierungen. Es werden Graphen und Knowledge Graphen genauer erläutert, ein Datenformat zur Speicherung der Daten und zuletzt die verschiedenen Visualisierungsoptionen. 

\section{Graphen und Knowledge Graphen}

\section{Resource Description Framework}

\section{Visualisierungstechniken}

Im Zeitalter der Digitalisierung sind Daten allgegenwärtig und werden in einem nie dagewesenen Ausmaß gesammelt. Es ist von entscheidender Bedeutung, diese Informationen effektiv begreifen und teilen zu können. In diesem Zusammenhang erweist sich die Datenvisualisierung als ein relevantes Feld, welches als Vermittler zwischen komplexen Datenmengen und dem menschlichen Verständnis fungiert. Durch die Transformation von rohen Daten in ein grafische Darstellungen vereinfacht die Datenvisualisierung den Zugang zu den Informationen. Sie erlaubt es den Beteiligten, Muster, Trends und Anomalien zu identifizieren, die in den Rohdaten verborgen sein könnten. Die Umwandlung von abstrakten Zahlen in visuelle Darstellungen fördert nicht nur das Verständnis, sondern unterstützt auch effektivere Entscheidungsprozesse. Es lässt sich nicht stark genug hervorheben, wie wichtig Datenvisualisierung ist. Sie macht Datenverständnis für alle zugänglich, ganz egal, wie viel oder wie wenig technisches Wissen der Benutzer hat (\cite{unwin:WhyDataVisualization}, S. 2 ff.).

In dem breiten Feld der Datenvisualisierung bildet die Visualisierung von Knowledge Graphen den Fokus der Untersuchungen in diesem Kapitel. Knowledge Graphen stellen nicht nur Daten dar, sie offenbaren ein komplexes Netzwerk von Verbindungen, das einzelnen Datenpunkten Kontext und Bedeutung verleiht (\cite{knowledgeGraph:whatIs}). Hierbei sind sowohl die Details der Verbindungen, als auch die übergeordnete Struktur von hoher Relevanz. Einerseits erlaubt die Betrachtung einzelner Knoten und Kanten einen detaillierten Einblick in die Beziehungen zwischen den Elementen. Andererseits erlaubt die Visualisierung der größeren Graphenstruktur, einen Blick auf die globale Topologie der Daten. Hier kann ein Verständnis für die übergeordneten Phänomene, wie Cluster, Hierarchien und Muster geschaffen werden, die aus den einzelnen Datenpunkten entstehen. Dieser doppelte Fokus bereichert das Verständnis der Daten und ermöglicht es, nicht nur die unmittelbaren Verbindungen, sondern auch die breiteren Muster und Strukturen zu erkennen. Auf diese Weise dient die Visualisierung von Knowledge Graphen als ein mächtiges Werkzeug, um die Komplexität vernetzter Informationen zu navigieren. Gleichzeitig bringt der doppelte Fokus allerdings auch Herausforderungen für jedes Werkzeug, welches die Visualisierung von Knowledge Graphen zum Ziel hat.

In den folgenden Abschnitten werden die möglichen Diagrammarten und Visualisierungstechniken, welche sich für die Darstellungen von Knowledge Graphen eignen detailliert beschrieben und erläutert.

\subsection{Diagrammarten}
\label{theory:visualization:types}

Bei der Analyse von Knowledge Graphen spielt die Visualisierung eine entscheidende Rolle. Sie fördert das Verständnis der komplexen Beziehungen zwischen den Entitäten und der Struktur der Daten. Nach der Analyse der möglichen Diagrammarten stechen Node-Link Diagramme und Adjazenzmatrizen als die vielversprechendsten Kandidaten für die Visualisierung heraus. Dieses Kapitel geht auf diese beiden  Diagrammtypen ein. Es werden ihre Eigenschaften hervorgehoben und ihre Vor- und Nachteile erläutert.

\subsubsection{Node-Link Diagramme}

Node-Link Diagramme sind die intuitivste Art der Graphdaten-Visualisierung. Sie bieten eine zugängliche und visuell  ansprechende Möglichkeit, die Vielschichtigkeit von Netzwerken darzustellen. In diesen Diagrammen werden Entitäten als Knoten abgebildet, die meistens durch Kreise dargestellt werden. Um unterschiedliche Typen von Entitäten oder deren Zustände zu kennzeichnen können allerdings auch andere Formen verwendet werden. Die Beziehungen zwischen diesen Entitäten werden durch Linien oder Pfeile dargestellt, wobei häufig verschiedene Stile oder Farben genutzt werden, um die Beschaffenheit oder Intensität dieser Verbindungen zu illustrieren. Die räumliche Positionierung der Knoten und Verbindungen in Node-Link Diagrammen kann die Lesbarkeit sowie die aus der Visualisierung gewonnenen Einsichten stark beeinflussen. Eine sinnvolle Anordnung der Knoten erlaubt es die spezifischen Charakteristiken des Netzwerks, wie beispielsweise Cluster zu betonen und diese erkennbar zu machen (\cite{nodelink:Basics}).

Der größte Vorteil von Node-Link Diagrammen besteht in ihrer Fähigkeit, abstrakte relationale Daten begreifbar zu machen. Sie ermöglichen es dem Betrachter Muster zu entdecken, Unregelmäßigkeiten zu erkennen und die Struktur des Netzwerks durch visuelle Untersuchung zu begreifen. Das macht sie unverzichtbar für jeden Einsatz, bei dem die Beziehungsdynamiken und die Struktur der Daten von Belangen sind. Ein Node-Link Diagramm kann beispielsweise in der Analyse sozialer Netzwerke auf einen Blick die Influencer aufzeigen, indem es die Knoten mit den meisten Verbindungen hervorhebt (\cite{nodelink:Basics}).

Allerdings nimmt die Effektivität von Node-Link Diagrammen mit der Größe und Komplexität des Graphen ab. In dicht vernetzten oder sehr großen Graphen kann die Übersichtlichkeit schnell abnehmen. Überlappende Verbindungen und Knoten führen oft zu einem Durcheinander, das die Visualisierung unklar und verwirrend macht. Dieses Problem kann durch interaktive Visualisierungstools reduziert werden. Diese Werkzeuge erlauben es Nutzern den Graphen dynamisch zu vergrößern, zu filtern und zu durchsuchen. Dennoch bleibt die Herausforderung der Verständlichkeit bei großen Graphen bestehen (\cite{nodelink:DynamicGraph}, S. 3). Dies erfordert eine sorgfältige Planung des Diagrammdesigns und den Einsatz von Vereinfachungstechniken, wie kraftbasierte Layouts und Kantenverschmelzung, um die Lesbarkeit zu verbessern.

To-do über Anordnung von Nodes schreiben

\subsubsection{Adjazenzmatrizen}

Einen gänzlich anderen Ansatz zur Visualisierung von Graphdaten bilden die Adjazenzmatrizen. Sie bieten eine strukturierte und analytische Perspektive auf die Daten. Adjazenzmatrizen verwenden ein rasterbasiertes Layout und eignen sich daher ausgezeichnet für die Abbildung komplexer und großflächiger Netzwerke. Bei einer Adjazenzmatrix werden die Entitäten des Graphen sowohl auf der horizontalen als auch auf der vertikalen Achse einer quadratischen Matrix aufgeführt. Die Zellen innerhalb der Matrix werden gefüllt, um die Existenz einer Beziehung zwischen den entsprechenden Entitäten zu signalisieren. Diese Herangehensweise verwandelt die Vernetzung des Graphen in ein visuelles Muster aus besetzten und unbesetzten Zellen. Eine besetzte Zelle am Kreuzungspunkt der Zeile ii und Spalte jj signalisiert eine direkte Verbindung von Entität ii zu Entität jj.

(Beispiel)

Die Stärke von Adjazenzmatrizen liegt in ihrer kompakten Darstellung großer und dichter Graphen. In solchen Fällen kann ein Node-Link Diagramm, das jeden Knoten und jede Kante visuell darstellt, schnell überladen und unlesbar werden. Im Gegensatz dazu nutzt eine Adjazenzmatrix den Raum optimal. Sie stellt die An- oder Abwesenheit von Kanten zwischen allen Knotenpaaren in einer Rasterstruktur dar. Diese Darstellung hat die Eigenschaft, dass sie keine Überlappungen oder Kreuzungen von Kanten erzeugt, was die Visualisierung deutlich übersichtlicher macht. Da sowohl vorhandene, als auch nicht vorhandene Verbindungen in der Matrix dargestellt werden, hängt die Größe der Matrix direkt mit der Anzahl der Knoten zusammen. Das bedeutet, dass unabhängig davon, wie dicht das Netzwerk wird, die Adjazenzmatrix eine konstante Größe beibehält. Diese Eigenschaft mach sie besonders für große Netzwerke geeignet, bei denen ein Node-Link Diagramm zu unübersichtlich wäre. 

Die Darstellung der Verbindungen in einer Matrix vereinfacht außerdem das Erkennen von Mustern. Sie ermöglicht eine klare und unmittelbare Visualisierung der Beziehungen zwischen Knoten, wodurch Muster und Cluster leichter erkannt werden können, als dies bei anderen Visualisierungstechniken der Fall ist. Diese Fähigkeit ist in der Datenanalyse besonders wertvoll, da sie eine effiziente und benutzerfreundliche Methode zur Identifizierung von Gruppenstrukturen in komplexen Netzwerken darstellt.

Die Darstellung in Matrixform hilft nicht nur bei der Visualisierung dichter Netzwerke, sondern erhöht auch die Recheneffizienz der Visualisierung. Bei einem Node-Link Diagramm wird viel Rechenleistung benötigt, um die optimale Positionierung der Knoten zu ermitteln. Das Ziel bei der Positionierung der Knoten in Node-Link Diagrammen ist, ein Diagramm zu erzeugen, dass eine hohe Lesbarkeit hat. Um das zu erreichen, wird versucht ein Layout zu erzeugen, bei dem die Kanten zwischen den Knoten so wenig wie möglich überlappen. Die Berechnung der optimalen Positionierung ist bei großen Netzwerken sehr rechenintensiv. Die Suche nach der optimalen Positionierung entfällt bei Adjazenzmatrizen gänzlich, da die Positionierung durch die Rasterstruktur vorgegeben wird und hier keine Überlappungen entstehen können.

Trotz aller Vorteile von Adjazenzmatrizen gibt es auch einen entscheidenden Nachteil. Adjazenzmatrizen sind für Personen, die nicht mit der Matrixdarstellung von Graphen vertraut sind, weniger zugänglich als Node-Link Diagramme. Das intuitive Verständnis, welches durch die räumliche Anordnung von Knoten und Kanten ermöglicht wird, fehlt hier, was es schwieriger macht, auf den ersten Blick ein intuitives Verständnis für die Gesamtstruktur des Graphen zu entwickeln.

Dennoch bieten Adjazenzmatrizen für bestimmte Aufgaben, wie den Vergleich struktureller Ähnlichkeiten zwischen zwei Graphen oder die Durchführung von detaillierten quantitativen Analysen, eine unübertroffene Klarheit und Effizienz. Um die Lesbarkeit und analytische Stärke von Adjazenzmatrizen zu verbessern, können verschiedene Techniken angewandt werden, wie etwa das Umordnen der Matrix, um ähnliche Entitäten nebeneinander zu platzieren, oder die Nutzung von Farbverläufen zur Darstellung unterschiedlicher Beziehungsintensitäten. Diese Methoden helfen, die Kluft zwischen struktureller Präsentation und intuitivem Verstehen zu schließen.

\subsubsection{Tree Maps}

Eine weitere Möglichkeit Graphen zu visualisieren sind Tree Map Diagramme. Sie erlauben es, hierarchische Daten durch das Einbetten von Rechtecken darzustellen. Jeder Ast des Baumes erhält ein Rechteck, das dann mit kleineren Rechtecken gefüllt wird, welche die Unteräste darstellen. Die Größe eines jeden Rechtecks kann dabei proportional zu einem spezifischen Datenwert sein, was eine direkte visuelle Korrelation zwischen der Größe des Rechtecks und der Bedeutung des dargestellten Elements ermöglicht. Tree Maps bieten eine äußerst effektive Methode, um Informationen über Proportionen innerhalb einer Hierarchie darzustellen. Dies macht sie besonders nützlich für die Visualisierung hierarchischer Daten, wie zum Beispiel Datei- und Verzeichnisstrukturen auf Computern.

Tree Map Diagramme bieten mehrere Vorteile bei der Datenvisualisierung, insbesondere in ihrer Fähigkeit, hierarchische Daten kompakt darzustellen und dabei den Raum effizient zu nutzen. Diese Eigenschaft macht sie zu einer ausgezeichneten Wahl für die Visualisierung komplexer Daten mit mehreren Hierarchieebenen in einer einzigen, leicht verständlichen Ansicht. Sie sind besonders wirksam beim Vergleichen der Größen von Kategorien und Unterkategorien innerhalb des Datensatzes. Tree Maps unterstützen außerdem eine Reihe von Anpassungsoptionen, wie Farbcodierungen und unterschiedliche Größen der Rechtecke, die dabei helfen können, Muster innerhalb der Daten hervorzuheben. All diese Eigenschaften machen sie zu einem ansprechenden Werkzeug für die Analyse hierarchischer Graphen und generell hierarchischer Daten.

Trotz ihrer Vorteile für die Visualisierung hierarchischer Daten stoßen Tree Map Diagramme auf erhebliche Einschränkungen, wenn sie auf große Knowledge Graphen angewendet werden. Knowledge Graphen sind Daten mit komplexen Verbindungen und Beziehungen, die über einfache hierarchische Strukturen hinausgehen. Sie verkörpern Netzwerke von Entitäten und deren Beziehungen zueinander. Oftmals umfassen sie große Mengen von untereinander komplex verbundenen Datenpunkten. Tree Map Diagramme sind konzeptionell nicht darauf ausgelegt, diese nicht-hierarchischen, komplexen Beziehungen darzustellen. Weiterhin nimmt mit steigender Größe und Komplexität des Graphen die Lesbarkeit von Tree Map Diagrammen signifikant ab, was sie für die Darstellung großangelegter, vernetzter Datensätze ungeeignet macht.

\subsection{Vergleiche der Diagrammarten}

Nachdem im vorangegangenen Kapitel Node-Link Diagramme und Adjazenzmatrizen ausführlich vorgestellt wurden, soll hier nun ein Vergleich folgen. In diesem Kapitel wird ein ausführlicher Vergleich von Node-Link Diagrammen mit Adjazenzmatrizen angestellt, um ihre Tauglichkeit für die Visualisierung großer Knowledge Graphen zu beurteilen. Der Vergleich erfolgt anhand folgender Kriterien: Lesbarkeit und Verständlichkeit, Skalierbarkeit, Interaktivität, Performance, Unterstützung für hierarchische Daten. Durch diesen Vergleich soll ein tiefes Verständnis für die Vor- und Nachteile jeder Visualisierungsart geschaffen werden, sodass später eine qualifizierte Auswahl einer Diagrammart getroffen werden kann. Die folgenden Punkte stützen sich auf die Erkenntnisse aus Kapitel \ref{theory:visualization:types}. Da sich die Tree Map Diagramme bereits in ihrer Vorstellung als ungeeignet gezeigt haben, werden diese im folgenden Kapitel nicht berücksichtigt.

\subsubsection{Lesbarkeit und Verständlichkeit}
\label{theory:visualization:comparison:readability}

Die Lesbarkeit und Verständlichkeit von Diagrammen sind entscheidende Faktoren für die Effektivität, mit der die Informationen an die Betrachter übermittelt werden können.

Node-Link Diagramme zeichnen sich durch ihre intuitive Darstellung von Netzwerkstrukturen aus. Sie repräsentieren Knoten als Punkte und Verbindungen als Linien, was eine direkte visuelle Erfassung der Netzwerkbeziehungen ermöglicht. Diese Art der Darstellung ist besonders vorteilhaft für die Visualisierung von kleinen bis mittelgroßen Netzwerken, da sie eine klare Übersicht über Verbindungen und Knotenpunkte bietet. Allerdings kann die Lesbarkeit bei sehr großen oder dichten Netzwerken abnehmen, da Überlappungen und Kreuzungen von Linien die Verständlichkeit erschweren.

Adjazenzmatrix Diagramme hingegen stellen Netzwerke in Form einer Matrix dar, wobei Zeilen und Spalten den Knoten entsprechen und die Präsenz oder Abwesenheit von Verbindungen durch Markierungen innerhalb der Matrix angezeigt wird. Diese Darstellungsform ist besonders effektiv für die Darstellung von dichter und großer Netzwerken und ermöglicht eine einfache Analyse von Beziehungen durch die Betrachtung der Matrixstruktur. Die Herausforderung bei Adjazenzmatrix Diagrammen liegt jedoch in der initialen Einarbeitung, da die Matrixdarstellung für ungeübte Betrachter weniger intuitiv ist, als die Darstellungen von Node-Link Diagrammen.

Zusammenfassend lässt sich sagen, dass die Wahl zwischen Node-Link und Adjazenzmatrix Diagrammen von der Größe und Dichte des Netzwerks sowie von den Zielen und der Zielgruppe der Visualisierung abhängt. Während Node-Link Diagramme durch ihre visuelle Darstellung eine höhere Verständlichkeit bieten, können Adjazenzmatrix Diagramme in der Analyse komplexer und dichter Netzwerke Vorteile aufweisen.

\subsubsection{Skalierbarkeit}
\label{theory:visualization:comparison:scalability}

Skalierbarkeit ist ein wesentliches Kriterium beim Vergleich von Diagrammtypen, insbesondere wenn es um die Visualisierung großer Knowledge Graphen geht. Im vorherigen Kapitel \ref{theory:visualization:comparison:readability} wurden bereit Aspekte der Skalierbarkeit angeschnitten. Dennoch lag der Fokus auf der allgemeinen Lesbarkeit der Diagrammtypen. Im Folgenden wird ein Vergleich von Node-Link Diagrammen und Adjazenzmatrizen hinsichtlich ihrer Skalierbarkeit vorgenommen. Das heißt, es soll die Frage beantwortet werden, wie gut lassen sich große Datenmengen darstellen und interpretieren.

Die Art der Darstellung von Node-Link Diagrammen ist besonders vorteilhaft für die Visualisierung kleiner bis mittelgroßer Netzwerke, allerdings stößt deren Ansatz bei großen oder dichten Netzwerken an seine Grenzen. Mit zunehmender Anzahl von Knoten und Verbindungen werden die Diagramme schnell überladen und unübersichtlich, was die Lesbarkeit und die Performance erheblich einschränkt.

Im Gegensatz dazu skaliert diese Darstellungsform von Adjazenzmatrizen besser mit der Größe des Netzwerks. Das liegt daran, dass sie unabhängig von der Anzahl der Knoten und Verbindungen eine gleichbleibend kompakte Darstellungsform bietet. Es müssen auch keine Kanten gezeichnet und geordnet werden, welche die Darstellung unübersichtlich machen und viel Rechenleistung für die optimale Positionierung benötigen.

Beim Vergleich der Skalierbarkeit von Node-Link und Adjazenzmatrix Diagrammen zeigt sich, dass Adjazenzmatrizen deutlich besser skalieren, als Node-Link Diagramme. Adjazenzmatrix-Diagramme bleiben auch bei großer Netzwerkgröße übersichtlich. Die Performance-Aspekte großer Datenmengen werden im nächsten Kapitel \ref{theory:visualization:comparison:performance} behandelt.

\subsubsection{Performance}
\label{theory:visualization:comparison:performance}

Das letzte Kriterium, was in diesem Vergleich beleuchtet werden soll, ist die Performance. Die Performance hätte bereits im vorangegangenen Kapitel \ref{theory:visualization:comparison:scalability} unter dem Gesichtspunkt der Skalierbarkeit betrachtet werden können. Allerdings wurde die bewusste Entscheidung getroffen eine Unterscheidung zwischen der allgemeinen Recheneffizienz der Diagrammarten und der Skalierbarkeit zu machen.

Node-Link Diagramme sind grafische Darstellungen, bei denen Knoten durch Punkte und Beziehungen durch Linien repräsentiert werden. Bei kleineren bis mittelgroßen Netzwerken ist die benötigte Rechenleistung für das Rendering und die Interaktion mit diesen Diagrammen logischerweise gering. Die Komplexität steigt jedoch exponentiell mit der Anzahl der Knoten und Verbindungen. Das reine Zeichnen der Knoten und Verbindungen stell hierbei allerdings weniger das Problem dar. Problematisch ist die Positionierung. Um die Übersichtlichkeit zu gewährleisten, werden in der Regel Algorithmen eingesetzt, welche versuchen die Knoten optimal zu positionieren, sodass möglichst wenig Überlappungen von Kanten entstehen. Die Performance wird auch von Techniken für die Vereinfachung der Diagramme belastet. Diese Techniken haben zu Ziel die Darstellung für den Nutzer übersichtlicher zu gestalten.  Diese Algorithmen benötigen bei großen und dichten Netzwerken erhebliche Rechenressourcen. Die Performance-Probleme manifestieren sich in Form von langsamen Ladezeiten und verzögerten Interaktionen.

Adjazenzmatrix Diagramme hingegen stellen Netzwerke als Matrizen dar, in denen die Verbindungen zwischen Knoten durch die Füllung der Zellen dargestellt wird. Die Rechenleistung für das Rendering einer Adjazenzmatrix skaliert im Wesentlichen linear mit der Anzahl der Knoten, da die Darstellung unabhängig von der Anzahl der Verbindungen ist. Dadurch sind Adjazenzmatrizen effizienter in der Darstellung sehr großer Netzwerke, insbesondere wenn es sich um dichte Netzwerke handelt. 

\section{Vereinfachungs- und Interaktionstechniken}

Im vorherigen Kapitel wurden die verschiedenen Diagrammarten, für die Visualisierung von Knowledge Graphen eingehend beleuchtet. Die Auswahl der richtigen Diagrammart ist entscheidend, um komplexe Zusammenhänge verständlich und übersichtlich darzustellen. Doch selbst mit der optimalen Diagrammart kann die Darstellung umfangreicher Knowledge Graphen schnell an Grenzen stoßen. Vor allem, wenn es darum geht, die Fülle an Informationen für den Betrachter zugänglich zu machen. In diesem Kapitel werden daher Techniken vorgestellt, die eingesetzt werden können, um diese Darstellungen zu vereinfachen und für den Betrachter zugänglicher zu machen.

\subsection{Node Clustering}

Node Clustering ist eine wichtige Technik bei der Vereinfachung von Node-Link Diagrammen. Die Technik zielt darauf ab, Gruppen, sogenannte Cluster, von Knoten innerhalb eines Graphen zu identifizieren. Die Elemente eines Clusters sind in irgendeiner Weise ähnlicher zueinander, als zum Rest des Graphen. Diese Ähnlichkeit kann sich auf verschiedene Aspekte beziehen, wie beispielsweise die Anzahl der Verbindungen oder die Werte der Knotenattribute. Node Clustering wird angewandt, um das visuelle Durcheinander eines Graphen zu minimieren und das menschliche Verständnis zu vertiefen (\cite{chen:SurveyGraphVisualization}, S. 634).

Das Identifizieren solcher Cluster ermöglicht es uns, tiefergehende Einblicke in die Struktur und Dynamik von Netzwerken zu gewinnen. Im folgenden Kapitel werden Algorithmen zur maschinellen Erkennung von Clustern, sowie Möglichkeiten zur Visualisierung von Clustern vorgestellt.