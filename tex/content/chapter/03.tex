%!TEX root = ../../main.tex

\chapter{Realisierung}

Nachdem im theoretischen Teil die Grundlagen, Herausforderungen und geeigneten Visualisierungstechniken für Knowledge Graphen erarbeitet wurden, widmet sich dieses Kapitel der praktischen Umsetzung der gewonnenen Erkenntnisse. Ziel ist es, ein System zu entwickeln, welches dazu in der Lage ist einen Knowledge Graphen zu visualisieren und den Anforderungen, die sich aus dem Theorieteil ableiten lassen, gerecht zu werden. Das System wird im Folgenden Kapitel als Graph Visualizer bezeichnet.

\section{Konzept}

In diesem Abschnitt wird das Konzept für die Umsetzung des Graph Visualizers entwickelt. Auf Basis des theoretischen Teils, werden zunächst die funktionalen und nicht-funktionalen Anforderungen an den Graph Visualizer analysiert. Anschließend werden die geeigneten Diagrammarten ausgewählt und die notwendigen Technologien festgelegt. Das erarbeitete Konzept bildet die Grundlage für die nachfolgende Implementierung und gewährleistet, dass der Graph Visualizer sowohl den technischen als auch den benutzerseitigen Anforderungen entspricht. Dabei wird besonders darauf geachtet, dass die Architektur des Systems flexibel und erweiterbar gestaltet ist, um zukünftige Anpassungen und Erweiterungen zu ermöglichen.

\subsection{Anforderungsanalyse}
\label{realization:requirements}

In diesem Abschnitt werden die spezifischen Anforderungen an den Graph Visualizer untersucht und definiert. Dabei wird zwischen funktionalen und nicht-funktionalen Anforderungen unterschieden. Die funktionalen Anforderungen beschreiben die konkreten Funktionen, die der Graph Visualizer erfüllen muss, wie etwa die Visualisierung von Knowledge Graphen, die Interaktivität und die Anpassungsfähigkeit der Darstellung. Die nicht-funktionalen Anforderungen hingegen befassen sich mit Aspekten wie Performance, Skalierbarkeit, Benutzerfreundlichkeit und Sicherheit. Die Anforderungen werden aus den Erkenntnissen aus Kapitel \ref{theory} abgeleitet. Diese Analyse bildet die Grundlage für die Auswahl der geeigneten Technologien und die Entwicklung eines Systems, das den Erwartungen der Benutzer entspricht und gleichzeitig den technischen Herausforderungen gerecht wird.

\subsubsection{Funktionale Anforderungen}

Für dieses Projekt wurde festgelegt, dass der Graph Visualizer die Daten des DBpedia Graphen visualisieren soll. DBpedia ist eine umfassende Wissensdatenbank, die auf strukturierten Informationen aus Wikipedia basiert. Diese Datenquelle wurde ausgewählt, da sie einen großen, frei zugänglichen Knowledge Graphen zur Verfügung stellt. Die genauen Daten sind in diesem Projekt zweitrangig. Es geht darum mit welchen Techniken ein Knowledge Graph dargestellt werden kann und nicht darum welche Daten dargestellt werden.

Der Graph Visualizer muss in der Lage sein, Knowledge Graphen effektiv darzustellen. Dies umfasst die Fähigkeit, Knoten und Kanten zu rendern und unterschiedliche Visualisierungstechniken anzuwenden, um die Daten strukturiert und verständlich zu präsentieren. Die Visualisierung muss es ermöglichen, komplexe Netzwerke übersichtlich darzustellen und verschiedene Perspektiven auf die Daten anzubieten.

Ein wesentliches Merkmal des Graph Visualizer ist die Interaktivität. Benutzer müssen die Möglichkeit haben, mit dem Graphen zu interagieren, indem sie Knoten und Kanten auswählen, verschieben, vergrößern oder verkleinern. Darüber hinaus soll der Benutzer durch einfache Aktionen, wie Klicks oder Drag-and-Drop, den Graphen dynamisch erkunden und manipulieren können, um tiefere Einblicke in die Daten zu gewinnen.

Der Graph Visualizer soll es ermöglichen, die Darstellung der Graphen anzupassen. Dazu gehört die Fähigkeit, verschiedene Layouts auszuwählen, sowie die Darstellung des Graphen anzupassen. Diese Anpassungen sollen helfen, die Visualisierung an den spezifischen Kontext der Analyse anzupassen und die relevanten Informationen hervorzuheben.

Die eben definierten funktionalen Anforderungen sind in der Tabelle \ref{tab:realization:concept:funcreq} zusammengefasst.

\begin{center}
    \begin{longtable}{|l|p{4.5cm}|p{8cm}|}
        \caption{Funktionale Anforderungen an den Graph Visualizer}
        \label{tab:realization:concept:funcreq}  \\

        \hline
        \multicolumn{1}{|c|}{}            &
        \multicolumn{1}{ c|}{\textbf{FR}} &
        \multicolumn{1}{ c|}{\textbf{Bedeutung}} \\
        \hline
        \endhead

        \hline
        \endlastfoot

        FR01
        \label{FR01}
                                          &
        Darstellung von Daten aus DBpedia
                                          &
        Das System soll DBpedia als Datenquelle verwenden.
        \\
        \hline
        FR02
        \label{FR02}
                                          &
        Visualisierung von Knowledge Graphen
                                          &
        Das System soll dazu in der Lage sein einen Knowledge Graphen zu visualisieren.
        \\
        \hline
        FR03
        \label{FR03}
                                          &
        Interaktivität
                                          &
        Das System muss dem Nutzer verschiedene Möglichkeiten zu Interaktion bieten.
        \\
        \hline
        FR04
        \label{FR04}
                                          &
        Anpassbarkeit der Darstellung
                                          &
        Der Nutzer muss die Darstellung anpassen können.
        \\
    \end{longtable}
\end{center}

\subsubsection{Nicht-Funktionale Anforderungen}

Der Graph Visualizer muss eine hohe Reaktionsgeschwindigkeit aufweisen, insbesondere bei der Verarbeitung großer Datenmengen. Ladezeiten sollten minimiert und Interaktionen wie das Verschieben, Zoomen oder Auswählen von Knoten sollten ohne Verzögerungen durchgeführt werden können. Das System sollte in der Lage sein, selbst bei Netzwerken mit tausenden von Knoten und Kanten flüssig zu arbeiten.

Das System muss stabil und zuverlässig funktionieren. Es darf nicht zu Abstürzen kommen, selbst wenn es mit unerwarteten Eingaben oder sehr großen Datenmengen konfrontiert wird.

Der Graph Visualizer sollte auf verschiedenen Plattformen und Betriebssystemen lauffähig sein. Der Nutzer soll den Graph Visualizer von jedem gängigen System aus verwenden können.

Das System sollte leicht wartbar sein. Der Quellcode muss gut strukturiert sein, um zukünftige Anpassungen und Erweiterungen zu erleichtern. Ein modularer Aufbau des Systems ist wünschenswert, um einzelne Komponenten einfach austauschen oder aktualisieren zu können.

Die Benutzeroberfläche sollte intuitiv und leicht verständlich sein. Es sollten umfassende Hilfsmittel wie Tooltips zur Verfügung stehen.

Die eben definierten nicht-funktionalen Anforderungen sind in der Tabelle \ref{tab:realization:concept:nonfuncreq} zusammengefasst.

\begin{center}
    \begin{longtable}{|l|p{4.5cm}|p{8cm}|}
        \caption{Nicht-funktionale Anforderungen an den Graph Visualizer}
        \label{tab:realization:concept:nonfuncreq} \\

        \hline
        \multicolumn{1}{|c|}{}             &
        \multicolumn{1}{ c|}{\textbf{NFR}} &
        \multicolumn{1}{ c|}{\textbf{Bedeutung}}   \\
        \hline
        \endhead

        \hline
        \endlastfoot

        NFR01
        \label{NFR01}
                                           &
        Performance
                                           &
        Die Anwendung muss eine hohe Reaktionsgeschwindigkeit aufweisen und selbst bei Graphen mit tausenden Knoten flüssig arbeiten.
        \\
        \hline
        NFR02
        \label{NFR02}
                                           &
        Zuverlässigkeit
                                           &
        Das System muss stabil und zuverlässig funktionieren.
        \\
        \hline
        NFR03
        \label{NFR03}
                                           &
        Kompatibilität
                                           &
        Der Graph Visualizer sollte auf verschiedenen Plattformen und Betriebssystemen lauffähig
        sein.
        \\
        \hline
        NFR04
        \label{NFR04}
                                           &
        Wartbarkeit
                                           &
        Das System sollte leicht zu warten und erweitern sein.
        \\
        \hline
        NFR05
        \label{NFR05}
                                           &
        Benutzerfreundlichkeit
                                           &
        Die Benutzeroberfläche sollte intuitiv und leicht verständlich sein.
        \\
    \end{longtable}
\end{center}

\subsection{Vergleiche und Auswahl der Visualisierungstechniken}

In diesem Kapitel werden die verschiedenen Visualisierungstechniken, die im theoretischen Teil der Arbeit vorgestellt wurden, analysiert und verglichen. Ziel ist es, die am besten geeigneten Techniken für die Implementierung im Graph Visualizer auszuwählen. Die Auswahl der Techniken erfolgt mit Blick auf die in Kapitel \ref{realization:requirements} definierten Anforderungen. Dieses Kapitel ist in zwei Unterkapitel unterteilt. Das erste beschäftigt sich mit der Auswahl geeigneter Diagrammarten, während das zweite die spezifischen Visualisierungstechniken betrachtet, die für die Darstellung der Knowledge Graphen eingesetzt werden sollen.

\subsubsection{Diagrammarten}

Für die Implementierung des Graph Visualizer werden sowohl Node-Link-Diagramme als auch Adjazenzmatrizen ausgewählt, da beide Diagrammarten unterschiedliche Stärken aufweisen, die den spezifischen Anforderungen des Projekts gerecht werden. Node-Link-Diagramme werden aufgrund ihrer intuitiven Darstellung und Benutzerfreundlichkeit gewählt, da sie es ermöglichen, die Beziehungen zwischen Knoten direkt und visuell ansprechend zu repräsentieren. Diese Diagrammart unterstützt die Anforderung, ein System zu entwickeln, das auch für Nutzer ohne tiefgehende technische Kenntnisse leicht verständlich und bedienbar ist. Auf der anderen Seite bieten Adjazenzmatrizen eine effiziente Möglichkeit, dichte Netzwerke kompakt und übersichtlich darzustellen, was insbesondere für die Anforderung der Skalierbarkeit entscheidend ist. Adjazenzmatrizen eignen sich hervorragend zur Analyse komplexer und stark verbundener Netzwerke, indem sie Verbindungen in einer klar strukturierten Form präsentieren. Obwohl beide Diagrammarten für die Implementierung ausgewählt wurden, spielen sie nicht zusammen, sondern werden getrennt verwendet, um ihre jeweiligen Vorteile auszunutzen. In einem späteren Kapitel werden sie miteinander verglichen, um zu evaluieren, welche der beiden Methoden für bestimmte Anwendungsfälle besser geeignet ist und wie sie sich in der praktischen Anwendung bewähren. Hierarchische Layouts hingegen werden nicht berücksichtigt, da ein Knowledge Graph typischerweise nicht hierarchisch ist und solche Layouts daher die komplexen, nicht-linearen Strukturen eines Knowledge Graphen nicht adäquat abbilden könnten.

\subsubsection{Visualisierungstechniken}

Für die Implementierung des Graph Visualizer wurden sorgfältig verschiedene Visualisierungstechniken ausgewählt, die den Anforderungen des Projekts am besten entsprechen und gleichzeitig die Grundlage für eine spätere Erweiterung des Systems bieten. Eine der zentralen Techniken ist die Graph Filterung, die sowohl in der Vorverarbeitung als auch während des Betriebs des Systems zum Einsatz kommt. Diese Technik ermöglicht es, irrelevante oder weniger wichtige Knoten und Kanten aus dem Graphen zu entfernen oder auszublenden, wodurch die Komplexität der Darstellung reduziert und der Fokus auf relevante Daten gelenkt wird. Dies ist besonders wichtig bei großen Netzwerken, da die Analyse solcher Netzwerke oft durch die schiere Menge an Informationen erschwert wird. Die interaktive Filterung, die es den Nutzern ermöglicht, den Graphen dynamisch an ihre Bedürfnisse anzupassen, trägt entscheidend zur Benutzerfreundlichkeit und Flexibilität des Systems bei. Zudem wird ein Force-Directed Layout implementiert, das sich durch seine intuitive Darstellung und die Möglichkeit der Anpassung aller Parameter durch den Nutzer auszeichnet. Diese Technik basiert auf physikalischen Modellen, die dafür sorgen, dass verwandte Knoten näher beieinanderliegen, was es dem Nutzer erleichtert, Cluster und Beziehungen im Netzwerk zu erkennen. Das Force-Directed Layout unterstützt die Anforderung, ein hochgradig interaktives und anpassbares System zu entwickeln, das auch komplexe Netzwerke verständlich darstellt. Ein weiterer wesentlicher Bestandteil des Graph Visualizer ist die Integration des Louvain-Algorithmus zur Clustererkennung. Dieser Algorithmus ist besonders für große Netzwerke geeignet und ermöglicht es, Cluster ohne vorherige Festlegung ihrer Anzahl zu identifizieren. Dies ist von besonderer Bedeutung für die Visualisierung von Knowledge Graphen, die in der Regel keine vorab bekannte Struktur aufweisen. Durch die Clustererkennung wird das Netzwerk in überschaubare Segmente unterteilt, was die Analyse erleichtert und die Skalierbarkeit des Systems verbessert. Andere im theoretischen Teil vorgestellte Techniken, wie das Edge Bundling oder die Fischaugen-Ansicht, werden aufgrund zeitlicher Einschränkungen nicht implementiert, obwohl sie ebenfalls nützlich sind. Der Fokus liegt auf den Methoden, die die Anforderungen am besten erfüllen und gleichzeitig die Grundlage für eine spätere Erweiterung des Systems bilden. Diese ausgewählten Techniken werden in späteren Kapiteln evaluiert, um ihre Effektivität und Eignung für die spezifischen Anforderungen des Graph Visualizer zu überprüfen und zu validieren.

\subsection{Technologieauswahl}
\subsubsection{Frontend-Technologien}
\subsubsection{Backend-Technologien}
\subsubsection{Schnittstellen und Datenformate}

\subsection{Erstellung des Lösungskonzepts}

\section{Implementierung}
\subsection{Umsetzung des Backends}
\subsection{Umsetzung des Frontends}

\section{Vergleich}
\label{realization:comparison}

\chapter{Zusammenfassung und Ausblick}