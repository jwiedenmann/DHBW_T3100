%!TEX root = ../../main.tex

\chapter{Realisierung}

Nachdem im theoretischen Teil die Grundlagen, Herausforderungen und geeigneten Visualisierungstechniken für Knowledge Graphen erarbeitet wurden, widmet sich dieses Kapitel der praktischen Umsetzung der gewonnenen Erkenntnisse. Ziel ist es, ein System zu entwickeln, welches dazu in der Lage ist einen Knowledge Graphen zu visualisieren und den Anforderungen, die sich aus dem Theorieteil ableiten lassen, gerecht zu werden. Das System wird im Folgenden Kapitel als Graph Visualizer bezeichnet.

\section{Konzept}

In diesem Abschnitt wird das Konzept für die Umsetzung des Graph Visualizers entwickelt. Auf Basis des theoretischen Teils, werden zunächst die funktionalen und nicht-funktionalen Anforderungen an den Graph Visualizer analysiert. Anschließend werden die geeigneten Diagrammarten ausgewählt und die notwendigen Technologien festgelegt. Das erarbeitete Konzept bildet die Grundlage für die nachfolgende Implementierung und gewährleistet, dass der Graph Visualizer sowohl den technischen als auch den benutzerseitigen Anforderungen entspricht. Dabei wird besonders darauf geachtet, dass die Architektur des Systems flexibel und erweiterbar gestaltet ist, um zukünftige Anpassungen und Erweiterungen zu ermöglichen.

\subsection{Anforderungsanalyse}

In diesem Abschnitt werden die spezifischen Anforderungen an den Graph Visualizer untersucht und definiert. Dabei wird zwischen funktionalen und nicht-funktionalen Anforderungen unterschieden. Die funktionalen Anforderungen beschreiben die konkreten Funktionen, die der Graph Visualizer erfüllen muss, wie etwa die Visualisierung von Knowledge Graphen, die Interaktivität und die Anpassungsfähigkeit der Darstellung. Die nicht-funktionalen Anforderungen hingegen befassen sich mit Aspekten wie Performance, Skalierbarkeit, Benutzerfreundlichkeit und Sicherheit. Die Anforderungen werden aus den Erkenntnissen aus Kapitel \ref{theory} abgeleitet. Diese Analyse bildet die Grundlage für die Auswahl der geeigneten Technologien und die Entwicklung eines Systems, das den Erwartungen der Benutzer entspricht und gleichzeitig den technischen Herausforderungen gerecht wird.

\subsubsection{Funktionale Anforderungen}

Für dieses Projekt wurde festgelegt, dass der Graph Visualizer die Daten des DBpedia Graphen visualisieren soll. DBpedia ist eine umfassende Wissensdatenbank, die auf strukturierten Informationen aus Wikipedia basiert. Diese Datenquelle wurde ausgewählt, da sie einen großen, frei zugänglichen Knowledge Graphen zur Verfügung stellt. Die genauen Daten sind in diesem Projekt zweitrangig. Es geht darum mit welchen Techniken ein Knowledge Graph dargestellt werden kann und nicht darum welche Daten dargestellt werden.

Der Graph Visualizer muss in der Lage sein, Knowledge Graphen effektiv darzustellen. Dies umfasst die Fähigkeit, Knoten und Kanten zu rendern und unterschiedliche Visualisierungstechniken anzuwenden, um die Daten strukturiert und verständlich zu präsentieren. Die Visualisierung muss es ermöglichen, komplexe Netzwerke übersichtlich darzustellen und verschiedene Perspektiven auf die Daten anzubieten.

Ein wesentliches Merkmal des Graph Visualizer ist die Interaktivität. Benutzer müssen die Möglichkeit haben, mit dem Graphen zu interagieren, indem sie Knoten und Kanten auswählen, verschieben, vergrößern oder verkleinern. Darüber hinaus soll der Benutzer durch einfache Aktionen, wie Klicks oder Drag-and-Drop, den Graphen dynamisch erkunden und manipulieren können, um tiefere Einblicke in die Daten zu gewinnen.

Der Graph Visualizer soll es ermöglichen, die Darstellung der Graphen anzupassen. Dazu gehört die Fähigkeit, verschiedene Layouts auszuwählen, sowie die Darstellung des Graphen anzupassen. Diese Anpassungen sollen helfen, die Visualisierung an den spezifischen Kontext der Analyse anzupassen und die relevanten Informationen hervorzuheben.

Die eben definierten funktionalen Anforderungen sind in der Tabelle \ref{tab:realization:concept:funcreq} zusammengefasst.

\begin{center}
    \begin{longtable}{|l|p{4.5cm}|p{8cm}|}
        \caption{Funktionale Anforderungen an den Graph Visualizer}
        \label{tab:realization:concept:funcreq}  \\

        \hline
        \multicolumn{1}{|c|}{}            &
        \multicolumn{1}{ c|}{\textbf{FR}} &
        \multicolumn{1}{ c|}{\textbf{Bedeutung}} \\
        \hline
        \endhead

        \hline
        \endlastfoot

        FR01
        \label{FR01}
                                          &
        Darstellung von Daten aus DBpedia
                                          &
        Das System soll DBpedia als Datenquelle verwenden.
        \\
        \hline
        FR02
        \label{FR02}
                                          &
        Visualisierung von Knowledge Graphen
                                          &
        Das System soll dazu in der Lage sein einen Knowledge Graphen zu visualisieren.
        \\
        \hline
        FR03
        \label{FR03}
                                          &
        Interaktivität
                                          &
        Das System muss dem Nutzer verschiedene Möglichkeiten zu Interaktion bieten.
        \\
        \hline
        FR04
        \label{FR04}
                                          &
        Anpassbarkeit der Darstellung
                                          &
        Der Nutzer muss die Darstellung anpassen können.
        \\
    \end{longtable}
\end{center}

\subsubsection{Nicht-Funktionale Anforderungen}

Der Graph Visualizer muss eine hohe Reaktionsgeschwindigkeit aufweisen, insbesondere bei der Verarbeitung großer Datenmengen. Ladezeiten sollten minimiert und Interaktionen wie das Verschieben, Zoomen oder Auswählen von Knoten sollten ohne Verzögerungen durchgeführt werden können. Das System sollte in der Lage sein, selbst bei Netzwerken mit tausenden von Knoten und Kanten flüssig zu arbeiten.

Das System muss stabil und zuverlässig funktionieren. Es darf nicht zu Abstürzen kommen, selbst wenn es mit unerwarteten Eingaben oder sehr großen Datenmengen konfrontiert wird.

Der Graph Visualizer sollte auf verschiedenen Plattformen und Betriebssystemen lauffähig sein. Der Nutzer soll den Graph Visualizer von jedem gängigen System aus verwenden können.

Das System sollte leicht wartbar sein. Der Quellcode muss gut strukturiert sein, um zukünftige Anpassungen und Erweiterungen zu erleichtern. Ein modularer Aufbau des Systems ist wünschenswert, um einzelne Komponenten einfach austauschen oder aktualisieren zu können.

Die Benutzeroberfläche sollte intuitiv und leicht verständlich sein. Es sollten umfassende Hilfsmittel wie Tooltips zur Verfügung stehen.

Die eben definierten nicht-funktionalen Anforderungen sind in der Tabelle \ref{tab:realization:concept:nonfuncreq} zusammengefasst.

\begin{center}
    \begin{longtable}{|l|p{4.5cm}|p{8cm}|}
        \caption{Nicht-funktionale Anforderungen an den Graph Visualizer}
        \label{tab:realization:concept:nonfuncreq} \\

        \hline
        \multicolumn{1}{|c|}{}             &
        \multicolumn{1}{ c|}{\textbf{NFR}} &
        \multicolumn{1}{ c|}{\textbf{Bedeutung}}   \\
        \hline
        \endhead

        \hline
        \endlastfoot

        NFR01
        \label{NFR01}
                                           &
        Performance
                                           &
        Die Anwendung muss eine hohe Reaktionsgeschwindigkeit aufweisen und selbst bei Graphen mit tausenden Knoten flüssig arbeiten.
        \\
        \hline
        NFR02
        \label{NFR02}
                                           &
        Zuverlässigkeit
                                           &
        Das System muss stabil und zuverlässig funktionieren.
        \\
        \hline
        NFR03
        \label{NFR03}
                                           &
        Kompatibilität
                                           &
        Der Graph Visualizer sollte auf verschiedenen Plattformen und Betriebssystemen lauffähig
        sein.
        \\
        \hline
        NFR04
        \label{NFR04}
                                           &
        Wartbarkeit
                                           &
        Das System sollte leicht zu warten und erweitern sein.
        \\
        \hline
        NFR05
        \label{NFR05}
                                           &
        Benutzerfreundlichkeit
                                           &
        Die Benutzeroberfläche sollte intuitiv und leicht verständlich sein.
        \\
    \end{longtable}
\end{center}

\subsection{Vergleiche und Auswahl der Visualisierungstechniken}

Nachdem im vorangegangenen Kapitel Node-Link Diagramme und Adjazenzmatrizen ausführlich vorgestellt wurden, soll hier nun ein Vergleich folgen. In diesem Kapitel wird ein ausführlicher Vergleich von Node-Link Diagrammen mit Adjazenzmatrizen angestellt, um ihre Tauglichkeit für die Visualisierung großer Knowledge Graphen zu beurteilen. Der Vergleich erfolgt anhand folgender Kriterien: Lesbarkeit und Verständlichkeit, Skalierbarkeit, Interaktivität, Performance, Unterstützung für hierarchische Daten. Durch diesen Vergleich soll ein tiefes Verständnis für die Vor- und Nachteile jeder Visualisierungsart geschaffen werden, sodass später eine qualifizierte Auswahl einer Diagrammart getroffen werden kann. Die folgenden Punkte stützen sich auf die Erkenntnisse aus Kapitel \ref{theory:visualization:types}. Da sich die Tree Map Diagramme bereits in ihrer Vorstellung als ungeeignet gezeigt haben, werden diese im folgenden Kapitel nicht berücksichtigt.

\subsubsection{Lesbarkeit und Verständlichkeit}
\label{theory:visualization:comparison:readability}

Die Lesbarkeit und Verständlichkeit von Diagrammen sind entscheidende Faktoren für die Effektivität, mit der die Informationen an die Betrachter übermittelt werden können.

Node-Link Diagramme zeichnen sich durch ihre intuitive Darstellung von Netzwerkstrukturen aus. Sie repräsentieren Knoten als Punkte und Verbindungen als Linien, was eine direkte visuelle Erfassung der Netzwerkbeziehungen ermöglicht. Diese Art der Darstellung ist besonders vorteilhaft für die Visualisierung von kleinen bis mittelgroßen Netzwerken, da sie eine klare Übersicht über Verbindungen und Knotenpunkte bietet. Allerdings kann die Lesbarkeit bei sehr großen oder dichten Netzwerken abnehmen, da Überlappungen und Kreuzungen von Linien die Verständlichkeit erschweren.

Adjazenzmatrix Diagramme hingegen stellen Netzwerke in Form einer Matrix dar, wobei Zeilen und Spalten den Knoten entsprechen und die Präsenz oder Abwesenheit von Verbindungen durch Markierungen innerhalb der Matrix angezeigt wird. Diese Darstellungsform ist besonders effektiv für die Darstellung von dichter und großer Netzwerken und ermöglicht eine einfache Analyse von Beziehungen durch die Betrachtung der Matrixstruktur. Die Herausforderung bei Adjazenzmatrix Diagrammen liegt jedoch in der initialen Einarbeitung, da die Matrixdarstellung für ungeübte Betrachter weniger intuitiv ist, als die Darstellungen von Node-Link Diagrammen.

Zusammenfassend lässt sich sagen, dass die Wahl zwischen Node-Link und Adjazenzmatrix Diagrammen von der Größe und Dichte des Netzwerks sowie von den Zielen und der Zielgruppe der Visualisierung abhängt. Während Node-Link Diagramme durch ihre visuelle Darstellung eine höhere Verständlichkeit bieten, können Adjazenzmatrix Diagramme in der Analyse komplexer und dichter Netzwerke Vorteile aufweisen.

\subsubsection{Skalierbarkeit}
\label{theory:visualization:comparison:scalability}

Skalierbarkeit ist ein wesentliches Kriterium beim Vergleich von Diagrammtypen, insbesondere wenn es um die Visualisierung großer Knowledge Graphen geht. Im vorherigen Kapitel \ref{theory:visualization:comparison:readability} wurden bereit Aspekte der Skalierbarkeit angeschnitten. Dennoch lag der Fokus auf der allgemeinen Lesbarkeit der Diagrammtypen. Im Folgenden wird ein Vergleich von Node-Link Diagrammen und Adjazenzmatrizen hinsichtlich ihrer Skalierbarkeit vorgenommen. Das heißt, es soll die Frage beantwortet werden, wie gut lassen sich große Datenmengen darstellen und interpretieren.

Die Art der Darstellung von Node-Link Diagrammen ist besonders vorteilhaft für die Visualisierung kleiner bis mittelgroßer Netzwerke, allerdings stößt deren Ansatz bei großen oder dichten Netzwerken an seine Grenzen. Mit zunehmender Anzahl von Knoten und Verbindungen werden die Diagramme schnell überladen und unübersichtlich, was die Lesbarkeit und die Performance erheblich einschränkt.

Im Gegensatz dazu skaliert diese Darstellungsform von Adjazenzmatrizen besser mit der Größe des Netzwerks. Das liegt daran, dass sie unabhängig von der Anzahl der Knoten und Verbindungen eine gleichbleibend kompakte Darstellungsform bietet. Es müssen auch keine Kanten gezeichnet und geordnet werden, welche die Darstellung unübersichtlich machen und viel Rechenleistung für die optimale Positionierung benötigen.

Beim Vergleich der Skalierbarkeit von Node-Link und Adjazenzmatrix Diagrammen zeigt sich, dass Adjazenzmatrizen deutlich besser skalieren, als Node-Link Diagramme. Adjazenzmatrix-Diagramme bleiben auch bei großer Netzwerkgröße übersichtlich. Die Performance-Aspekte großer Datenmengen werden im nächsten Kapitel \ref{theory:visualization:comparison:performance} behandelt.

\subsubsection{Performance}
\label{theory:visualization:comparison:performance}

Das letzte Kriterium, was in diesem Vergleich beleuchtet werden soll, ist die Performance. Die Performance hätte bereits im vorangegangenen Kapitel \ref{theory:visualization:comparison:scalability} unter dem Gesichtspunkt der Skalierbarkeit betrachtet werden können. Allerdings wurde die bewusste Entscheidung getroffen eine Unterscheidung zwischen der allgemeinen Recheneffizienz der Diagrammarten und der Skalierbarkeit zu machen.

Node-Link Diagramme sind grafische Darstellungen, bei denen Knoten durch Punkte und Beziehungen durch Linien repräsentiert werden. Bei kleineren bis mittelgroßen Netzwerken ist die benötigte Rechenleistung für das Rendering und die Interaktion mit diesen Diagrammen logischerweise gering. Die Komplexität steigt jedoch exponentiell mit der Anzahl der Knoten und Verbindungen. Das reine Zeichnen der Knoten und Verbindungen stell hierbei allerdings weniger das Problem dar. Problematisch ist die Positionierung. Um die Übersichtlichkeit zu gewährleisten, werden in der Regel Algorithmen eingesetzt, welche versuchen die Knoten optimal zu positionieren, sodass möglichst wenig Überlappungen von Kanten entstehen. Die Performance wird auch von Techniken für die Vereinfachung der Diagramme belastet. Diese Techniken haben zu Ziel die Darstellung für den Nutzer übersichtlicher zu gestalten.  Diese Algorithmen benötigen bei großen und dichten Netzwerken erhebliche Rechenressourcen. Die Performance-Probleme manifestieren sich in Form von langsamen Ladezeiten und verzögerten Interaktionen.

Adjazenzmatrix Diagramme hingegen stellen Netzwerke als Matrizen dar, in denen die Verbindungen zwischen Knoten durch die Füllung der Zellen dargestellt wird. Die Rechenleistung für das Rendering einer Adjazenzmatrix skaliert im Wesentlichen linear mit der Anzahl der Knoten, da die Darstellung unabhängig von der Anzahl der Verbindungen ist. Dadurch sind Adjazenzmatrizen effizienter in der Darstellung sehr großer Netzwerke, insbesondere wenn es sich um dichte Netzwerke handelt.

\subsection{Technologieauswahl}
\subsubsection{Frontend-Technologien}
\subsubsection{Backend-Technologien}
\subsubsection{Schnittstellen und Datenformate}

\subsection{Erstellung des Lösungskonzepts}

\section{Implementierung}
\subsection{Umsetzung des Backends}
\subsection{Umsetzung des Frontends}

\chapter{Zusammenfassung und Ausblick}