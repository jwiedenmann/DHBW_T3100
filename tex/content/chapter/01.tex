%!TEX root = ../../main.tex

\chapter{Einleitung}

Die Visualisierung von Daten ist zu einem großen Feld in der modernen Forschung herangewachsen. Die Disziplin beschäftigt sich mit der Erstellung von leicht verständlichen grafischen Darstellungen aus großen mengen komplexer Daten. Sie ermöglicht es, komplexe Informationen auf eine zugängliche Weise zu präsentieren und tiefergehende Erkenntnisse aus den vorliegenden Daten zu gewinnen (vgl. \cite{wiki:DataVisualization}). Grafische Darstellungen können Datenmerkmale erkennbar machen, welche in rohen Statistiken verloren gehen können (vgl. \cite{unwin:WhyDataVisualization}, S. 2).

In den vergangenen Jahren hat sich die Datenlandschaft allerdings stark gewandelt. Die Bedeutung von verknüpften Informationen hat im Zeitalter von Big Data stark zugenommen. In immer mehr Feldern und Anwendungen fallen große Mengen von Daten an, welche die Beziehung zwischen Entitäten beschreiben. Zu diesen Feldern zählen beispielsweise Biomedizin, soziale Netzwerke und Kommunikationsnetzwerke (vgl. \cite{chen:SurveyGraphVisualization}, S. 1 f). Diese Daten lassen sich in Form von Graphen beschreiben. Ein Graph ist ein mathematisches Konzept, welches mithilfe einer Menge von Knoten und Kanten, Objekte und deren Beziehungen untereinander modellieren kann (vgl. \cite{wiki:Graph}).

Graphen lassen sich mathematisch gut beschreiben. Allerdings wird für die visuelle Analyse der Daten eine grafische Darstellung benötigt. Es existieren zahlreiche Diagramme, mit welchen Graphen dargestellt werden können. Beispielsweise gibt es Node-Link, Adjazenzmatrix und Hypergraph Diagramme (vgl. \cite{chen:SurveyGraphVisualization}, S. 2). Jede dieser Diagrammklassen ist dazu in der Lage einen Graphen darzustellen. Dabei sind sie allerdings nicht jeder Aufgabe gleich gut gewachsen. Jede der möglichen Diagrammklassen folgt einem anderen Ansatz und hat unterschiedliche Prioritäten in der Darstellung. So sind Node-Link-Diagramme beispielsweise besonders leicht zu verstehen, werden allerdings schnell unübersichtlich, wenn die Anzahl der Objekte oder der Kanten steigt. Das Problem, bei der Darstellung von Graphen liegt also nicht in dem Mangel von adäquaten Visualisierungstechniken. Im Gegensatz liegt es in der Auswahl der richtigen Technik. Aufgrund der unterschiedlichen Prioritäten, bei der Visualisierung, ist die Auswahl ist allerdings maßgeblich von dem vorliegenden Problem abhängig.

\section{Motivation}
\section{Zielsetzung}

Ziel der Arbeit ist es zu ermitteln, welche Visualisierungstechnik sich am besten für Knowledge Graphen eignet. Anschließend soll die Visualisierung eines Knowledge Graphen implementiert werden.

Aufgrund der unterschiedlichen stärken, wird zum Vergleich der verschiedenen Visualisierungstechniken ein konkretes Beispiel benötigt. Diese Arbeit, wurde auf Knowledge Graphen eingeschränkt. Knowledge Graphen haben heutzutage eine hohe Relevanz. Sie werden traditionell in Suchmaschinen und in Sprachassistenten, wie Siri, eingesetzt. Zu den klassischen Anwendungsfällen sind in den letzten Jahren noch die Neuralen Netze hinzugekommen (vgl. \cite{wiki:KnowledgeGraph}).

Zunächst wird ein Überblick über die möglichen Visualisierungstechniken geliefert. Diese werden ausführlich beschrieben und ihre Vor- und Nachteile aufgezeigt. Anschließend werden die Techniken verglichen und die für das Problem geeignetste ausgewählt. Im praktischen Teil der Arbeit wird dann die ausgewählte Visualisierungstechnik implementiert. 

\section{Methodik}

Im theoretischen Teil der Arbeit werden durch Literaturrecherche die möglichen Visualisierungstechniken und deren Vor- und Nachteile ermittelt. Diese werden dann verglichen und eine Auswahl getroffen.

Im praktischen Teil der Arbeit wird die ausgewählte Visualisierungstechnik dann implementiert. Das Frontend wird über JavaScript realisiert. Hierbei kommt die Bibliothek D3 zum Einsatz, um den Graphen darzustellen. Das Backend wird mittels C\# realisiert werden. Die Daten für den Knowledge Graphen werden aus der DBpedia API abgerufen. Die API stellt Wikipedia Informationen, als Knowledge Graph bereit. Der Endpunkt wird über die Abfragesprache \ac{SPARQL} abgefragt.


% (vgl. \cite{unwin:WhyDataVisualization}, S. 2)
% (vgl. \cite{wiki:DataVisualization})
% (vgl. \cite{wiki:Graph})
% (vgl. \cite{chen:SurveyGraphVisualization})
