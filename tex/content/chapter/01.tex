%!TEX root = ../../main.tex

\chapter{Einleitung}

Die Visualisierung von Daten hat sich in den letzten Jahren zu einem bedeutenden Forschungsfeld entwickelt. Sie befasst sich mit der Erstellung verständlicher grafischer Darstellungen aus komplexen und großen Datenmengen. Dadurch wird es möglich, komplexe Informationen auf eine zugängliche Weise darzustellen und tiefere Erkenntnisse aus den Daten zu gewinnen. Grafische Darstellungen können Muster und Merkmale sichtbar machen, die in rohen Statistiken oft übersehen werden (vgl. \cite{unwin:WhyDataVisualization}, S. 2).

In den letzten Jahren hat sich die Datenlandschaft jedoch drastisch verändert, insbesondere durch den Aufstieg von Big Data. Verknüpfte Informationen, die die Beziehungen zwischen Entitäten beschreiben, haben enorm an Bedeutung gewonnen. Besonders in Bereichen wie der Biomedizin, sozialen Netzwerken und Kommunikationsnetzwerken fallen große Mengen solcher Daten an (vgl. \cite{chen:SurveyGraphVisualization}, S. 1 f.). Diese Daten können durch Graphen modelliert werden. Sie sind mathematische Strukturen, die Knoten und Kanten verwendet, um komplexe Netzwerke darzustellen.

Obwohl Graphen mathematisch gut beschrieben werden können, erfordert die visuelle Analyse der Daten eine geeignete grafische Darstellung. Es existieren zahlreiche Visualisierungstechniken, die Graphen darstellen können, darunter Node-Link-Diagramme, Adjazenzmatrizen und Hypergraphen (vgl. \cite{chen:SurveyGraphVisualization}, S. 2). Jede dieser Methoden hat jedoch ihre spezifischen Stärken und Schwächen. So sind Node-Link-Diagramme beispielsweise leicht verständlich, werden jedoch schnell unübersichtlich, wenn die Anzahl der Knoten oder Kanten zunimmt. Das eigentliche Problem bei der Visualisierung von Graphen liegt nicht im Fehlen geeigneter Techniken, sondern in der Auswahl der richtigen Methode. Diese Auswahl hängt stark vom jeweiligen Anwendungsfall ab, da jede Technik unterschiedliche Aspekte der Daten hervorhebt.

\section{Zielsetzung}

Ziel der Arbeit ist es zu ermitteln, welche Visualisierungstechnik sich am besten für Knowledge Graphen eignet. Anschließend soll die Visualisierung eines Knowledge Graphen implementiert werden.

\section{Aufbau der Arbeit}

Die vorliegende Arbeit gliedert sich in vier Hauptkapitel, die systematisch aufeinander aufbauen und die verschiedenen Aspekte der Visualisierung von großen Datasets aus Graph-Datenbanken behandeln.

Im ersten Kapitel, der Einleitung, wird das Thema der Arbeit eingeführt und die Zielsetzung klar definiert. Es wird erläutert, welche Relevanz die Visualisierung von Graph-Datenbanken hat, insbesondere im Kontext großer und komplexer Datenmengen. Darüber hinaus wird das Ziel der Arbeit formuliert.

Das zweite Kapitel beschäftigt sich mit den theoretischen Grundlagen der Arbeit. Es erläutert die Definition und Merkmale von Graphen sowie Knowledge Graphen und geht auf deren Unterschiede ein. Weiterhin werden Anwendungsgebiete von Knowledge Graphen beschrieben und verschiedene Visualisierungstools wie Gephi und Cytoscape vorgestellt.

Das dritte Kapitel behandelt die praktische Umsetzung der im theoretischen Teil gewonnenen Erkenntnisse. Es beginnt mit einer detaillierten Anforderungsanalyse, in der sowohl funktionale als auch nicht-funktionale Anforderungen an die entwickelte Lösung definiert werden. Anschließend wird die Auswahl der verwendeten Technologien sowie die konkrete Systemarchitektur beschrieben. Der Hauptteil dieses Kapitels widmet sich der Implementierung.

Im vierten und letzten Kapitel erfolgt eine Zusammenfassung der Ergebnisse der Arbeit. Die wichtigsten Erkenntnisse werden zusammengetragen und kritisch bewertet, wobei sowohl die Stärken als auch die Schwächen der entwickelten Lösung beleuchtet werden. Abschließend gibt das Kapitel einen Ausblick auf mögliche Weiterentwicklungen und Erweiterungen der Visualisierungstechniken sowie die potenziellen Einsatzmöglichkeiten der entwickelten Lösung in weiteren Anwendungsbereichen.